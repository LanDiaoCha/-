\section{二次函数的形式与图像性质}

\subsection{顶点式}

\(y=a(x-h)^2+k\)是定点式的形式,其中\(k\)和\(h\)的值控制着函数平移的方向和距离.


\begin{figure}[h]
    \centering
    \begin{tikzpicture}[scale=0.9]
    \begin{axis}[
        axis lines = middle,       % 坐标轴居中
        > = Stealth,
        xlabel = $x$,             % x轴标签
        ylabel = $y$,             % y轴标签
        xmin = -5, xmax = 5,      % x轴范围
        ymin = -1, ymax = 7,      % y轴范围
        xtick = {-4,-3,...,4},    % x轴刻度
        ytick = {-1,...,1,2,3,4,5,6},  % y轴刻度
        legend pos = north west,  % 图例位置
    ]
    
    \addplot [
        domain = -5:6,
        samples = 100,
        thick,
        gray,
        densely dashed
    ] {x^2};
    
    \node[anchor=west, gray] at (axis cs:3,5) {\(y=x^2\)};

     \addplot [
        domain = -5:6,
        samples = 100,
        thick,
        blue
    ] {x^2+3};
    

    \addplot [
        domain = -6:6,
        samples = 100,
        thick,
        red
    ] {(x+3)^2+3};
    
    \node[anchor=west, red] at (axis cs:-4,6) {\(\)};

    \draw[densely dashed, ultra thick, blue] (0,0) -- (0,3) node[anchor=west, blue] at (axis cs:0.3,2) {$y=x^2+3$};

    \draw[densely dashed, ultra thick, red] (0,3) -- (-3,3) node[anchor=south, red] at (axis cs:-2.7,2) {$y=(x-(-3))^2+3$};
    
    \end{axis}
    \end{tikzpicture}
    \caption{}
    \label{fuc_2}
\end{figure}

如上图所示,
%顶点式的关键是顶点坐标为(h,k)

\subsection{一般式}

顶点式实际上由一般形式配方而来, 二次函数的顶点式常用于直观表现出二次函数的顶点, 一般形式主要体现抛物线与系数的关系

\subsection{交点式(两根式)}

