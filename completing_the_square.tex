\section{配方}



配方是一种把二次多项式(如 \(ax^2 + bx + c\))通过添加和减去同一个项,转化为完全平方式的数学技巧.
\par
\begin{remark}
    完全平方是指一个数或代数式可以表示为另一个数或代数式的平方,例如\( 1=1^2,\ 4=2^2,\ \)(这些是数的完全平方)\(x^2+6x+9=(x+3)^2,\ 4x^2-12xy+9y^2=(2x-3y)^2\)等等.
\end{remark}
\par
在八年级我们学习过完全平方公式\(a^2+b^2\pm2ab=(a\pm b)^2\)来因式分解,这个公式本质上就是在配方. 但并不是所有式子都能使用这个公式来配方,例如\(x^2+2x\)就没有办法直接套用公式,面对这种情况就需要使用配方法来转化为一个有完全平方的式子,形如\(a(x+h)^2+k\),下面就用\(x^2+2x\)作为配方示范:
\begin{example}
给\(x^2+2x\)配方
\end{example}
\par
\begin{solution}
    \begin{enumerate}
  \item 分析一次项结构,将一次项分解为:\\
  \[
  2x = 2 \cdot x \cdot 1
  \]
  这对应完全平方公式中的 \(2ab\),其中 \(a = x\),\(b = 1\)
  
  确定需要添加的项,根据完全平方公式,需要添加 \(b^2\):
  \[
  b^2 = (1)^2 = 1
  \]
  
  \item 同时增减此项
  \begin{align*}
  x^2 + 2x &= x^2 + 2x + 1 - 1 \\
            &= (x^2 + 2x + 1) - 1
  \end{align*}
  
  \item 重组为完全平方式,前三项恰好是 \(x\) 和 \(1\) 的完全平方
  \begin{remark}
  将$1^2$移出括号不能忘记乘上括号前的系数1)
  \end{remark}
  \[
  = (x + 1)^2-1
  \]
  \item 一般来说最后需要整理方程,$(x + 1)^2-1$已经最简所以不用整理
\end{enumerate}

\end{solution}
以上是平方项(二次项)为1的情况,当平方项为1时,才能分解一次想的结构从而配出另一个平方项,但是当平方项不为1呢?这个时候就要先把平方项化为1,可以通过“因式分解-提取公因式”做到,下面以给\(2x^2 + 6x+1\)配方为例子
\par
%%%%%%%%%%%%%%%%%%%%%%%%%%%%%%%%%%%%%%%%%%%%%%%%%%%%%%%%%%%%%%%%%%%%%%%%%%%%%%
\begin{example}
    将 \(2x^2 + 6x+1\) 配方
\end{example}
\begin{solution}
\begin{enumerate}
    \item 提取前两项的系数 \(2\):
    \begin{align*}
    \text{原式} &= 2\left(x^2 + 3x\right) + 1
    \end{align*}

    \item 在括号内配方(补全平方):
    \begin{align*}
    &= 2\left[x^2 + 3x + \left(\frac{3}{2}\right)^2 - \left(\frac{3}{2}\right)^2\right] + 1\\
    &=2\left[\left(x + \frac{3}{2}\right)^2 - \frac{9}{4}\right] + 1
    \end{align*}

    \item 展开并合并常数项
    \begin{remark}
    将\(\frac{4}{9}\)移出括号不能忘记乘上提取的系数)
    \end{remark}
    \begin{align*}
    &=2\left(x + \frac{3}{2}\right)^2 - \frac{9}{2} + 1 \\
    &= 2\left(x + \frac{3}{2}\right)^2 - \frac{7}{2}
    \end{align*}
\end{enumerate}

最终配方结果为:
\[
2\left(x + \frac{3}{2}\right)^2 - \frac{7}{2}
\]

\end{solution}

以上两道是已知一个平方项和一次项配另一个平方项的例题,你是否发现我们配的平方项和一次项系数之间的关系?
\par
当已知平方项的系数为1,设一次项系数为\(A\),那我们配上的项就是\((\dfrac{A}{2})^2\)(一次项系数的一半的平方),运用这个关系在配方时就不用再拆解一次项进行分析,但要注意必须在平方项系数为1时,所以在配方时先检查平方项系数是否为1,如果不是可以像例题1.6一样提取公因式把平方项系数化为1再配方,对于已知一个平方项和一次项配另一个平方项的题型,总结配方步骤如下:
\begin{enumerate}
    \item 把平方项(二次项)系数化为1
    \item 同时增加和减少一次项系数的一半的平方
    \item 将要配的三项写成完全平方形式(把要减去的项提出来时要乘上括号前的系数)
    \item 整理式子成一般形式:\( a(x + h)^2 + k \)
\end{enumerate}
\par
配方不仅能够因式分解,还可以解决一些其他的数学问题,例如配方得到多项式的极值,配方后开方降次解一元二次方程. 这一节先讲用配方得到式子的极值
\par
由于平方具有非负性,也就是说\(a^2\geq0\),且\(a=0\)时,\(a^2\)的最小值为0,根据这一点,
当二次多项式配方为 \( a(x + h)^2 + k \) 时,因为 \( (x + h)^2 \geq 0 \),所以:
\newline
\begin{itemize}[label=]
    \item 如果 \( a > 0 \),多项式的最小值为 \( k \)(当 \( x = -h \) 时取得);
    \item 如果 \( a < 0 \),多项式的最大值为 \( k \)(当 \( x = -h \) 时取得).
\end{itemize}

\begin{example}
若\(x\)为任意实数,求代数式\(x^2+4x+2\)的最小值
\end{example}

\begin{solution}

\begin{align*}
x^2 + 4x + 2 &= x^2 + 4x + 4 - 4 + 2 \\
&= (x^2 + 4x + 4) - 2 \\
&= (x + 2)^2 - 2.\\
\end{align*}
\begin{align*}
&\because (x + 2)^2 \geq 0\\
&\therefore (x + 2)^2-2 \geq -2\\
&\therefore x^2 + 4x + 2 \  \text{的最小值是}-2
\end{align*}
\end{solution}

\begin{exercise}
\small
    \setlength{\parindent}{0pt} % 取消段落缩进
    \setlength{\columnseprule}{0.01pt}
    \begin{multicols}{2}
        \begin{minipage}{1\linewidth}
        (1)给\( 4x^2 +1\)配上一个单项式,使之成为完全平方式.
        \end{minipage}
        
        \begin{minipage}{1\linewidth}
        (2)给下列多项式配方.
        \begin{enumerate}
            \item \((2x+1)^2-2(2x+1)+1\)
            \item \(3x^2-6x-2\)
            \item \(3x^2+8x-3\)
            \item \(x^2+2x+10\)
        \end{enumerate}
        \end{minipage}
%%%%%%%%%%%%%%%%%%%%%%%%%%%%%%%%%%%%%%%%%%%%%%%%%%%%%%%%%%%还有习题没编完,参考5星学霸第六页
        \begin{minipage}{1\linewidth}
        (3)\(-x^2+4x-7\)的最小值是\underline{\hspace{3.5em}},此时\(x=\)\underline{\hspace{3.5em}}
        \end{minipage}

        \begin{minipage}{1\linewidth}
        (4)若\(W=5x^2-4xy+y^2-2y+8x+3(x,y\text{为实数}),\text{则}W\)的最小值为\underline{\hspace{3.5em}}
        \end{minipage}

        (5)已知\(M=x^2+2y^2, N=2xy+y-m\),当\(M>N\)恒成立时,求\(m\)的取值范围.
        
    \end{multicols}
\end{exercise}



%配方的意义重大->如何配方->配方看取值->配方分解因式

%配方法解方程->用配方发推导出公式法->根的判别式->求根公式解方程->因式分解


