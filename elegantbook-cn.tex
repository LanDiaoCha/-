\documentclass[lang=cn, 10pt, titlestyle=hang]{elegantbook}

\title{九年级数学讲义(人教版)}


\begin{document}

\tableofcontents

\chapter{一元二次方程}



\begin{introduction}[本章学习目标]

\item 将一元二次方程整理为一般形式
\item 判定一元二次方程
\item 熟练掌握配方法
\item 使用配方法求最值
\item 用不同方式解一元二次方程

\end{introduction}

经过本章的学习,你至少需要掌握以上的内容,在完成学习后你可以根据习题的情况,对照上面的学习目标,检验自己的学习成果.

\section{认识一元二次方程(定义)}
等号两边都是整式,只含一个未知数,并且未知数的最高次数是2的方程,叫做一元二次方程. 以下是二元一次方程的一般形式:

$$
ax^2 + bx + c = 0 \ (a \neq 0)
$$



我们将$ax^2$称为二次项(未知数的次数为2),a是二次项的系数;$bx$称为一次项(未知数的次数为1),\\ $b$是一次项系数;$c$称为常数项.



$x$的值就是这个一元二次方程的解,$x$的值也可以被称作这个一元二次方程的根,在做题时,有时会把$x$的值称做方程的解,有时称作方程的根,其实这两种说法都是一个意思.

\section{整理一元二次方程}

在解题时,我们通常会先将一元二次方程整理为一般形式($
ax^2 + bx + c = 0 \ (a \neq 0)
$),这样做可以减少我们的工作量和出错的概率,便于进行计算.



整理过程分两步:
\begin{enumerate}
    \item 移项、合并
    \item 将二次项的系数化为非负数
\end{enumerate}

\section{判定一元二次方程}

在1.1中我们学习了一元二次方程的定义,在1.2中我们学习了如何整理一元二次方程,上面的两节课虽然简单,却是本章最重要的内容,下面是测试题,可以用较短时间完成,如果你还未掌握,请回看前面的两节课再继续.



经过总结,我们可以得到步骤:



\begin{enumerate}
    \item 整理方程
    \item 判断是否符合条件
\end{enumerate}


\begin{problemset}[1.1-1.3习题]
    \item 下列关于$x$的方程中,是一元二次方程的为($\ \ \ \ \ \ \ \ \ $) \\
    A. $x^2 + \dfrac{1}{x} = 0$\\
    B. $x^2 - xy = 0$\\
    C. $x^2 + \dfrac{1}{x} = 0$\\
    D. $x^2 + \dfrac{1}{x} = 0$\\
\end{problemset}

\section{配方法}



\section{解一元二次方程}

\chapter{二次函数}

\chapter{旋转}

\chapter{圆}

\chapter{概率初步}

\chapter{反比例函数}

\chapter{相似}

\chapter{锐角三角函数}

\chapter{投影与视图}

\end{document}