\section{二次函数与坐标轴、直线的交点}

本节探讨二次函数与坐标轴或任意直线的交点有关的问题.分为两类,一类是与线段、射线或直线的交点,另一类是与坐标轴的交点
\subsection{与线段、射线或直线的交点}
这类题目通常会提供一个含参的二次函数和一个范围(区间、直线、坐标轴或其他),然后探讨参数取何值能使二次函数与这个范围有一个交点、两个交点或没有交点.我们首先探讨二次函数与横轴的交点,以此为基础再推广到抛物线与任何直线的交点.
\subsubsection*{代数思想}

%%%%%%%%%%%%%%%%%%%%%%%%%%%%%%%%%%%%%%%%%%%%函数中的定值
%横轴上一点,二次项系数开口向下,一定有两个交点

%定点在横轴上,必定只有一个交点

%函数里的k必须能被“抵消”掉(即含k的式子能等于 0)时,函数过定点

%---------------------------------------开始分类


首先要观察表达式中提供的信息,例如二次函数过某一定点、开口方向、对称轴等等信息,利用已知信息会减少下来不必要的分类讨论.

要想把二次函数限定在横轴上要从四个方面考虑:
\begin{enumerate}
    \item \textbf{开口方向}(这是分类的前提条件)
    \item \textbf{判别式}(限定根的情况)
    \item \textbf{对称轴}(根的值不确定时,限定范围)
    \item \textbf{范围端点}(紧接对称轴限定的范围继续限定在范围的两个端点内)
\end{enumerate}

下面通过例题实际感受:

\begin{example}
    已知抛物线 \( y = x^2 - kx - k \),\( A (1, -2) \),\( B (4, 10) \).抛物线与线段 \( AB \)(包括 \( A \),\( B \) 两点)有两个交点,则 \( k \) 的取值范围为\underline{\hspace{4.5em}}.
\end{example}

\begin{example}
已知抛物线 \( y_1 = mx^2 + 4x - 2 \) (\( m \neq 0 \)) 的顶点为 \( C \).已知点 \( M(-1, -4), N(3, 0) \),若该抛物线与线段 \( MN \) 总有两个不同的交点,请求出 \( m \) 的取值范围.
\end{example}

%1确定开口方向
%+
%有无定量(过定点)
%可以得出
%横轴上下一点,二次项系数开口向下或向下,一定有两个交点


%------------------------------------------卡交点

在确定开口方向的情况下234

2确定判别式的情况(限定在横轴上有无交点/有几个交点)
二次函数与横轴的交点的横坐标实际上是\(y=0\)时\(x\)的值,可以看作是一个一元二次方程,要判断这个二次函数与横轴有没有交点.实际上就是判断一元二次方程有没有实数根,使用根的判别式\(\Delta\)就能解决.结合"求根公式法"这一章就知道如果有一个交点那么\(\Delta=0\),如果有两个交点,则\(\Delta>0\)

%-------------------------------------------卡范围


3确定对称轴在题目给定的范围内


4范围两端点




恒等变形转化为与横轴的交点






\subsubsection*{图像思想}

前面我们使用代数思想分析交点类的问题,使用代数法解决交点问题比较通用.但在遇到已知条件较为充足时可以采用数形结合的思想快速解决此类问题,适用于条件丰富无需分类讨论的题目,缺点是不能适用于所有题目,比较局限.

\begin{example}
一段抛物线 \( L: y = -x (x-3) + c \)\( (0 \leq x \leq 3) \) 与直线 \( l: y = x + 2 \) 有唯一公共点,则 \( c \) 的取值范围为\underline{\hspace{4em}}。
\end{example}

\subsection{与坐标轴的交点}
明确交点类型
分类讨论函数类型
分析交点总数
特殊情形处理

检查函数是否通过原点(0,0)

检查与y轴交点是否同时也是与x轴交点
